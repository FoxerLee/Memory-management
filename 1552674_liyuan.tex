\documentclass{article}
\usepackage{CJK}
\usepackage{indentfirst}
\usepackage{anysize}
\usepackage{graphicx}
\usepackage{subfigure}
\usepackage{array}
\usepackage{makecell}
\usepackage{url}
\usepackage{float}

%标题缩进
%\usepackage[bf, small]{titlesec}
%    \titleformat{\section}{\bf\large}{\thesection.\,}{0.24em}{}
%    \titlespacing{\section}{0cm}{*1.5}{*1.1}
%    \titleformat{\subsection}{\bf}{\thesubsection.\enspace}{0.5em}{}
%    \titlespacing{\subsection}{15pt}{*1.5}{*1.1}
%    \titleformat{\subsubsection}{}{\thesubsubsection.\,}{0.24em}{}
%    \titlespacing{\subsubsection}{30pt}{*1.5}{*1.1}

\linespread{1.5}


\marginsize{3.5cm}{3.5cm}{2cm}{2cm}
\setlength{\parindent}{2em}

\newcolumntype{L}[1]{>{\vspace{0.5em}\begin{minipage}{#1}\raggedright\let\newline\\
\arraybackslash\hspace{0pt}}m{#1}<{\end{minipage}\vspace{0.5em}}}
\newcolumntype{R}[1]{>{\vspace{0.5em}\begin{minipage}{#1}\raggedleft\let\newline\\
\arraybackslash\hspace{0pt}}m{#1}<{\end{minipage}\vspace{0.5em}}}
\newcolumntype{C}[1]{>{\vspace{0.5em}\begin{minipage}{#1}\centering\let\newline\\
\arraybackslash\hspace{0pt}}m{#1}<{\end{minipage}\vspace{0.5em}}}


%使得图片显示对应章节
\renewcommand\thefigure{\thesection.\arabic{figure}}
\makeatletter
\@addtoreset{figure}{section}
\makeatother

\begin{document} 
\begin{CJK}{UTF8}{gbsn}

\newcommand*{\titleGP}{\begingroup % Create the command for including the title page in the document
\centering % Center all text
\vspace*{\baselineskip} % White space at the top of the page

\rule{\textwidth}{1.6pt}\vspace*{-\baselineskip}\vspace*{2pt} % Thick horizontal line
\rule{\textwidth}{0.4pt}\\[\baselineskip] % Thin horizontal line

{\LARGE 请求调页存储管理方式模拟
 \\ \vspace{2em} \begin{large} 操作系统课程设计 \end{large}}\\[0.2\baselineskip] % Title

\rule{\textwidth}{0.4pt}\vspace*{-\baselineskip}\vspace{3.2pt} % Thin horizontal line
\rule{\textwidth}{1.6pt}\\[\baselineskip] % Thick horizontal line

\scshape % Small caps
%利用操作系统中的多线程思想,自我实现电梯调度算法 \\[\baselineskip] % Tagline(s) or further description
operating system,  Spring 2017\par % Location and year

\vspace*{2\baselineskip} % Whitespace between location/year and editors

 By \\[\baselineskip]
{\Large1552674 李源 \par} % Editor list


\vfill % Whitespace between editor names and publisher logo

{\itshape Tongji University \\ School of Software Engineering \par}

\endgroup}


\titleGP % This command includes the title page
\clearpage
\tableofcontents
\clearpage

\section{项目背景}
\subsection{项目需求}
假设每个页面可存放10条指令,分配给一个作业的内存块为4。模拟一个作业的执行过程,该作业有320条指令,即它的地址空间为32页,目前所有页还没有调入内存。

\subsection{项目目的}
\begin{itemize}
	\setlength{\itemsep}{0.5pt}
	\item 掌握页面、页表、地址转换过程;
	\item 对页面置换过程有更深的认识;
	\item 加深对请求调页系统的原理和实现过程的理解。
\end{itemize}

\vspace{3em}

\section{需求分析}
根据项目需求,我们可以得知本项目所模拟的内存和作业分别满足如下要求。结合内存和作业的实际情况,我们可以设计出本项目的模拟过程。

\begin{itemize}
	\setlength{\itemsep}{0.5pt}
	\item 内存:4个内存块,一个内存块中能存放10条指令;
	\item 作业:320条指令,分别放在32页中。
\end{itemize}



\subsection{模拟过程}
在模拟过程中,对于每一条待访问的指令,存在下属两种可能:
\begin{itemize}
	\setlength{\itemsep}{0.5pt}
	\item 该指令在内存中,则显示其物理地址,并转到下一条指令;
	\item 该指令不在内存中,则发生缺页,此时需要记录缺页次数,并将其调入内存。
\end{itemize}

\vspace{3em}

\section{调度算法}

\vspace{3em}
\section{系统实现}


\vspace{3em}
\section{开发环境}
\begin{itemize}
	\item 系统:macOS Sierra (version 10.12.4)
	\item IDE:Qt Creator 4.2.1, Based on Qt 5.8.0 (Clang 7.0 (Apple), 64 bit)
	\item 语言: C++
\end{itemize}

\vspace{3em}
\section{提交内容}
\begin{itemize}
	\item 源代码
	\item assignment2.zip 可执行文件压缩包(需在mac系统下使用)
	\item assignment2.dmg 安装包(需在mac系统下使用)
	\item 说明文档
	\item 演示视频
\end{itemize}

\end{CJK}
\end{document}